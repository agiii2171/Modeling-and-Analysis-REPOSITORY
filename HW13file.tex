%%%%%%%%%%%%%%%%%%%%%%%%%%%%%%%%%%%%%%%%%%%%%%%%%%%%%%%%%%%%
%%%%%%%%%%%%%%%%%%%%%%%%%%%%%%%%%%%%%%%%%%%%%%%%%%%%%%%%%%%%
%%%%%%%%%%%%%%%%%%%%%%%%%%%%%%%%%%%%%%%%%%%%%%%%%%%%%%%%%%%%
%%%%%%%%%%%%%%%%%%%%%%%%%%%%%%%%%%%%%%%%%%%%%%%%%%%%%%%%%%%%
%%%%%%%%%%%%%%%%%%%%%%%%%%%%%%%%%%%%%%%%%%%%%%%%%%%%%%%%%%%%
\documentclass[12pt]{article}
\usepackage{epsfig}
\usepackage{times}
\usepackage{mathabx}
\usepackage{amsmath}
\usepackage[dvipsnames]{xcolor}
\renewcommand{\topfraction}{1.0}
\renewcommand{\bottomfraction}{1.0}
\renewcommand{\textfraction}{0.0}
\setlength {\textwidth}{6.6in}
\hoffset=-1.0in
\oddsidemargin=1.00in
\marginparsep=0.0in
\marginparwidth=0.0in                                                                               
\setlength {\textheight}{9.0in}
\voffset=-1.00in
\topmargin=1.0in
\headheight=0.0in
\headsep=0.00in
\footskip=0.50in                                         
\setcounter{page}{1}
\begin{document}
\def\pos{\medskip\quad}
\def\subpos{\smallskip \qquad}
\newfont{\nice}{cmr12 scaled 1250}
\newfont{\name}{cmr12 scaled 1080}
\newfont{\swell}{cmbx12 scaled 800}
%%%%%%%%%%%%%%%%%%%%%%%%%%%%%%%%%%%%%%%%%%%%%%%%%%%%%%%%%%%%
%     DO NOT CHANGE ANYTHING ABOVE THIS LINE
%%%%%%%%%%%%%%%%%%%%%%%%%%%%%%%%%%%%%%%%%%%%%%%%%%%%%%%%%%%%
%     DO NOT CHANGE ANYTHING ABOVE THIS LINE
%%%%%%%%%%%%%%%%%%%%%%%%%%%%%%%%%%%%%%%%%%%%%%%%%%%%%%%%%%%%
%     DO NOT CHANGE ANYTHING ABOVE THIS LINE
%%%%%%%%%%%%%%%%%%%%%%%%%%%%%%%%%%%%%%%%%%%%%%%%%%%%%%%%%%%%

\begin{center}
{\large
PHYSICS  20323: Scientific Analysis \& Modeling - Fall 2023
}\\
%%%%%%%%%%%%%%%%%%%%%%%%%%%%%%%%%%%%%%%%%%%%%%%%%%%%%%%%%%%%
{\large Project: Anthony Gerg}\\\vskip0.25in
%%%%%%%%%%%%%%%%%%%%%%%%%%%%%%%%%%%%%%%%%%%%%%%%%%%%%%%%%%%%
\end{center}
%%%%%%%%%%%%%%%%%%%%%%%%%%%%%%%%%%%%%%%%%%%%%%%%%%%%%%%%%%%%
% Section Heading
%%%%%%%%%%%%%%%%%%%%%%%%%%%%%%%%%%%%%%%%%%%%%%%%%%%%%%%%%%%%
\begin{enumerate}
    \item {\bf The following questions refer to the stars in the table below} \\
    Note: there may be multiple answers
\begin{center}
\begin{tabular}{|c|c|c|c|c|c|}\hline
Name & Mass & Luminosity & Lifetime & Temperature & Radius  \\\hline
$\eta$ Car.  & $60.$ $M_{\Sun}$ & $10^6$ $L_{\Sun}$ & $8.0 \times 10^5$ years &  &  \\\hline
$\epsilon$ Eri. & $6.0$ $M_{\Sun}$ & $10^3$ $L_{\Sun}$ &  & $20,000$ K &   \\\hline
$\sigma$ Scu. & $2.0$ $M_{\Sun}$ &  & $5.0 \times 10^8$ years &  & $2$ $R_{\Sun}$ \\\hline
$\beta$ Cyg.  & $1.3$ $M_{\Sun}$ & $3.5$ $L_{\Sun}$ &  &  &   \\\hline
$\alpha$ Cen. & $1.0$ $M_{\Sun}$ &  &  &  & $1$ $R_{\Sun}$ \\\hline
$\gamma$ Del. & $0.7$ $M_{\Sun}$ &  & $4.5 \times 10^10$ years & $5000$ K &   \\\hline
\end{tabular}\vskip 0.2in
\end{center}
\begin{itemize}
    \item[(a)] (4 points) Which of these stars will produce a planetary nebula.
    \vskip0.35in
    \item[(b)] (4 points) Elements heavier than \textit{Carbon} will be produced in which stars.
\end{itemize}
\item An electron is found to be in the spin state (in the $z$-basis): \boldmath{$\chi$$=$} A 
$\begin{pmatrix}
$3i$ \\
$4$ 
\end{pmatrix}$
\begin{itemize}
    \item[(a)] (5 points) Determine the values of A such that the state is normalized
    \vskip0.35in
    \item[(b)] (5 points) Find the expectation values of \textit{\textcolor{red}{$S_x$}}, \textit{\textcolor{purple}{$S_y$}}, \textit{\textcolor{Orange}{$S_z$}}, and \textit{$\vec{S^2}$}
\end{itemize}
\vskip0.5in
The matrix representation in the $z$-basis for the matrix for the components of the electrons spin operators are given by: \\
\begin{equation*}
\textit{\textcolor{red}{$S_x = \frac{\hbar}{2} $\begin{pmatrix}
$0$ & $1$ \\
$1$ & $0$
\end{pmatrix}$ $;}}
\hspace{0.2in}
\textit{\textcolor{purple}{$S_y = \frac{\hbar}{2} $\begin{pmatrix}
$0$ & $-i$ \\
$i$ & $0$
\end{pmatrix}$ $;}}
\hspace{0.2in}
\textit{\textcolor{Orange}{$S_z = \frac{\hbar}{2} $\begin{pmatrix}
$0$ & $-i$ \\
$i$ & $0$
\end{pmatrix}$ $;}}
\end{equation*}
\item The average electrostatic field in Earth's atmosphere in fair weather is given by:
\begin{equation}
\vec{E} = E_o (Ae^{-\alpha z} + Be^{-\beta z}) \hat{z}
\end{equation}
where A, B, $\alpha$, $\beta$ are positive constants and $z$ it the height above the (locally flat) earth surface.
\vskip0.2in
\begin{itemize}
    \item[(a)] (5 points) Find the average charge density in the atmosphere as a function of height.
    \vskip0.35in
    \item[(b)] (5 points) Find the electric potential as a function height above the earth.
\end{itemize}
\end{enumerate}
\end{document}